\documentclass{article}

\usepackage[a4paper, landscape, margin=5pt]{geometry}
\usepackage{hyperref}       % links
\usepackage{multicol}       % for columns layout
\usepackage{blindtext}      % lorem ipsum
\usepackage{listings}       % code
\usepackage{xcolor}          % color

% settings
\setlength{\parindent}{0pt}     % no first row paragraph indentation
\setlength{\parskip}{0pt}       % Space between paragraphs
\setlength{\multicolsep}{0pt}   % no bottom and top space in multicols
\setlength{\columnsep}{2pt}     % column separation
\pagestyle{empty}               % no page numbers
% setting listings to Racket style
\lstdefinestyle{Haskell}{
    aboveskip=0pt,
    belowskip=0pt,
    commentstyle=\color{orange},
    basicstyle=\ttfamily\color{darkgray},
    keywordstyle=\ttfamily\bfseries\color{black},
    stringstyle=\color{blue},       
    breaklines=true,                              
    keepspaces=true,
    language=Haskell,
    mathescape=true,
    alsoletter={!, ?},
    keywords={data, map, show, module, let, in, zip, where, class, false, instance, take, not, deriving, id}
}
\lstset{style=Haskell}
\newcommand{\dollar}{\mbox{\textdollar}}

\begin{document}
\textbf{Haskell CheatSheet} - \href{https://github.com/isthissuperuser}{https://github.com/isthissuperuser} \textbar{ }  \textbf{legenda}: \{ \} optional, [ ] one or more, 
\begin{multicols*}{3}

\hrulefill

\textbf{MISC}
\begin{lstlisting}
data name = food | fish | .. -- sum/union
data name a = name {getterX, getterYa :: a} -- constructor takes 2 arg of same type a and also expose already the getters
let var1 = value1
    var2 = value2
in  body
$\dollar$ -- evaluate before right expression
show -- print
zip lst1 lst2 -- couples from each lst
module filename (fnames) where -- export functions in module
\end{lstlisting}

\hrulefill

\textbf{CONTROL FLOW}
\begin{lstlisting}
\end{lstlisting}

\hrulefill

\textbf{VECTORS}
\begin{lstlisting}
\end{lstlisting}

\hrulefill

\textbf{LISTS}
\begin{lstlisting}
[a] -- list of variable type
[cur:rest]
take 3 lst -- return first 3 elems 
lst [1, 3 .. 10] -- [1, 3, 6, 9]
lst [ x * 2 | x <- [0, 1 ..]] -- even numbers
++ -- list concat
\end{lstlisting}

\hrulefill

\textbf{FUNCTIONS}
\begin{lstlisting}
fname arg = body
map (f1 . f2) lst -- first map f2 then f1
(\ x -> x * 2) -- lambda
fname arg@(cur:rest) = body -- arg is string decomposed
fname num
 | f1 < 2 = "small"
 | f1 > 2 = "big"
 where
    f1 = fbody
id -- identity function
4 `div` 2 -- integer division
\end{lstlisting}

\hrulefill

\textbf{CLASSES}
\begin{lstlisting}
class name parameter where
    functions -- create classes
class Equal a where
    (==) :: a -> a -> bool
    x /= y = not (x == y)
data Tree a = Leaf a | Branch (Tree a) (Tree a)
instance (Equal a) => Equal (Tree a) where
    Leaf a == Leaf b = a == b
    (Branch a1 b1) == (Branch a2 b2) = a1 == a2 && b1 == b2
    _ == _ = False
-- instantiating Equal class with a Tree as arg
-- and implementing (==) func for it
data Tree a = Leaf a | Branch (Tree a) (Tree a) deriving (Eq) -- deriving for class inheritance 
instance Foldable Tree where -- Foldable has no args
    foldr :: (a -> b -> b) -> b -> Tree a -> b
    -- foldr func acc Tree returns acc
-- instance is for pseudo classes inheritance
\end{lstlisting}

\hrulefill

\textbf{FUNCTORS}
\begin{lstlisting}
data Result a = Err | Ok a deriving (Eq, Ord, Show)
instance Functor Result where
    fmap f (Ok a) = Ok $\dollar$ f a
    fmap f Err = Err
-- fmap can be called also with <$\dollar$>
-- functors are for applying functions of one argument to a particular object
\end{lstlisting}

\hrulefill

\textbf{APPLICATIVES}
\begin{lstlisting}
instance Applicative Result where
    (<*>) :: Result (a -> b) -> Result a -> Result b
    (Ok f) <*> (Ok x) = Ok $\dollar$ f x
    _ <*> Err = Err
    Err <*> _ = Err

    pure = Ok
-- applicative are like Functors but they work with 2 parameters and the first one can also be a partially applied function
[(+1), (*2), (^3)] <*> [1,2,3]
[2, 3, 4, 2, 4, 6, 1, 8, 27]
-- partial f applied to list and concat
\end{lstlisting}

\hrulefill

\textbf{PATTERNS}
\begin{lstlisting}
\end{lstlisting}

\end{multicols*}
\end{document}
